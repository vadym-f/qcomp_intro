% Vadym Fedyukovych  2020
% You are free to Share and Adapt Under the following terms: Attribution.
% Attribution 4.0 International (CC BY 4.0)
%
\documentclass[14pt]{beamer}

\usepackage{cmap}
\usepackage{droid} % XCharter tempora PTSans
\usepackage[T2A]{fontenc}
\usepackage[utf8]{inputenc}
\usepackage[russian,english]{babel}

\usepackage{amssymb,amsmath,amsthm,url}
\usepackage{listings}

\usepackage{tikz}
\usetikzlibrary{quantikz}

\author{Вадим Федюкович}
\title{Квантовые вычисления \\
  практическое ведение \\
  \small{\url{https://github.com/vadym-f/qcomp_intro}}}
\date{15 октября 2020}

\usepackage{hyperref}
\hypersetup{
  pdfauthor={Vadym Fedyukovych},
%  pdftitle={QuantumcomputingIntro},
%  pdftitle={\@title}
  pdfkeywords={Licensed CC BY 4.0},
  colorlinks=true,
  urlcolor=blue,
  unicode
}

\begin{document}
\frame  {
\titlepage
}

\frame {
  \frametitle{Аксиомы и ожидаемые знания}
  \begin{itemize}
  \item
      векторы, базис, амплитуды, тензорное произведение
  \item
      комплексное сопряжение, скалярное произведение, вероятность, Гильбертово пространство
  \item
      операторы $=$ матрицы, собственные векторы и значения
  \item
      измерение и 'разрушение' исходного состояния, проектор
  \item
      суперпозиция и 'вычисления' одновременно на всех состояниях
  \end{itemize}

  {\small \href{http://www.mathnet.ru/conf1820}
               {Математические основы квантовой криптографии,
                А.С.Трушечкин, Д.А.Кронберг, МИАН}}
% Bell/CHSH
}

\frame {
  \frametitle{Модели вычислений}
  \begin{itemize}
  \item
      'Гейты' и компьютер с Джозефсоновскими кубитами
  \item
      Поиск минимального состояния, \href{https://arxiv.org/pdf/1302.5843.pdf}{D-Wave}
  \item
      Гауссовы состояния фотонов, \href{https://www.xanadu.ai/research}{Xanadu}
  \item
      'Топологические' вычисления с косами/braid, \href{https://www.microsoft.com/en-us/quantum/microsoft-quantum-labs}{anion/majorana Microsoft}
  \item
      Коммуникации и 'квантовый Интернет': модели 'вычислительного центра' и 'мобилка для каждого'
  \end{itemize}

  {\small \href{https://arxiv.org/abs/1904.06560}
               {A Quantum Engineer's Guide to Superconducting Qubits}}
}

\frame {
  \frametitle{``Таблица умножения''}
  \begin{gather*}
    \ket{0} = \begin{bmatrix} 1 \\ 0 \end{bmatrix},  \quad
    \ket{1} = \begin{bmatrix} 0 \\ 1 \end{bmatrix}
\\
    X = \begin{bmatrix} 0 & 1 \\ 1 & 0 \end{bmatrix},  \quad
    Z = \begin{bmatrix} 1 & 0 \\ 0 & -1 \end{bmatrix},  \quad
    H = \frac{1}{\sqrt{2}} \begin{bmatrix} 1 & 1 \\ 1 & -1 \end{bmatrix}
\\    
    X \ket{0} = \ket{1},  \;
    X \ket{1} = \ket{0}
\quad
    Z \ket{0} = \ket{0},  \;
    Z \ket{1} = -\ket{1}
\\
    H \ket{0} = \frac{\ket{0} + \ket{1}}{\sqrt{2}},  \quad
    H \ket{1} = \frac{\ket{0} - \ket{1}}{\sqrt{2}}
  \end{gather*}
  \mbox{$CNOT$ (на диаграмме $\oplus$ и 'управляющий' кубит):} \\
  \mbox{если 'control' $ = \ket{1}$ применяет $X$ к 'target'-кубиту.}
\\
  Состояние двух двухуровневых систем и тензорное произведение: $\ket{00}_{AB} = \ket{0}_A \otimes \ket{0}_B$
\\
{\small
  \href{https://biblio.mccme.ru/node/1547}
       {Классические и квантовые вычисления}
}
}
% KAU summer school 2018

\frame {
  \frametitle{Задача Дойча}

  \begin{quantikz}
    \lstick{\ket{0}} & \gate{H}     & \gate[wires=2][2cm]{f()}
                                      \gateinput{$x$}
                                      \gateoutput{$x$}
    \slice{}    & \gate{H} & \meter{} & \cw
    \begin{cases}
       \text{constant:} 0 \\
       \text{balanced:} 1
    \end{cases}
\\
    \lstick{\ket{1}} & \gate{H}     &
                                      \gateinput{$y$}
                                      \gateoutput{$y \oplus f(x)$}
                & \qw      &&
  \end{quantikz}
  \begin{gather*}
%    1: \;
%   \frac{\ket{0} + \ket{1}}{\sqrt{2}} \otimes \frac{\ket{0} - \ket{1}}{\sqrt{2}} =
    (\ket{0} + \ket{1}) \otimes (\ket{0} - \ket{1}) =
    \ket{00} + \ket{10} - \ket{01} - \ket{11}
\\
%    2: \;
    \ket{00 \oplus f(0)} + \ket{10 \oplus f(1)} - \ket{01 \oplus f(0)} - \ket{11 \oplus f(1)} =
\\
    \ket{0, f(0)} + \ket{1, f(1)} - \ket{0, \overline{f(0)}} - \ket{1, \overline{f(1)}} =
\\
    \ket{0} \otimes (\ket{f(0)} - \ket{\overline{f(0)}}) +
    \ket{1} \otimes (\ket{f(1)} - \ket{\overline{f(1)}}) =
\\
    (\ket{0} \pm \ket{1}) \otimes (\ket{f_0} - \ket{\overline f_0})
%    \begin{cases}
%       \text{f() is constant: } (\ket{0} + \ket{1}) \otimes (\ket{f} - \ket{\bar f})\\
%       \text{f() is balanced: } (\ket{0} - \ket{1}) \otimes ()
%    \end{cases}
  \end{gather*}
}

\frame {
  \frametitle{Телепортация -- 1}

  \begin{quantikz}
    \lstick{\ket{\psi}} & \qw \slice{$A_1$} & \qw \slice{$A_2$} & \ctrl{1} \slice{$A_3$} & \gate{H} \slice{$A_4$} & \meter{$M_1$} \slice{$B$} & \cw & \cwbend{2} & \rstick{Initial}
    \\
    \lstick{\ket{0}} &   \gate{H} & \ctrl{1}  &  \targ{} & \qw & \meter{$M_2$} & \cwbend{1} &&&&
    \\
    \lstick{\ket{0}} & \qw & \targ{} & \qw & \qw & \qw & \gate{X} & \gate{Z} & \rstick{New} \qw
  \end{quantikz}

  \begin{gather*}
    \ket{\psi} \otimes \frac{\ket{0} + \ket{1}}{\sqrt{2}} \otimes \ket{0},  \quad
    (\alpha \ket{0} + \beta \ket{1}) \otimes \frac{\ket{00} + \ket{11}}{\sqrt{2}}
\\
    A_3: \;
    \alpha \ket{0} \otimes \frac{\ket{00} + \ket{11}}{\sqrt{2}} +
    \beta \ket{1} \otimes \frac{\ket{10} + \ket{01}}{\sqrt{2}}
  \end{gather*}
}

\frame {
  \frametitle{Телепортация -- 2}

  \begin{gather*}
%    A_4: \;
    \frac{\alpha (\ket{0} + \ket{1})(\ket{00} + \ket{11}) +
          \beta (\ket{0} - \ket{1})(\ket{10} + \ket{01})  }{2}
  \end{gather*}

\begin{tabular}{ | c | c | c | c | }
\hline
 $M_1$ $M_2$    &  $B$                 & Action  &  Result  \\
\hline
 00  & $\alpha \ket{0} + \beta \ket{1}$ &  $1$  & $\alpha \ket{0} + \beta \ket{1}$ \\ 
 01  & $\alpha \ket{1} + \beta \ket{0}$ &  $X$  & $\alpha \ket{0} + \beta \ket{1}$ \\  
 10  & $\alpha \ket{0} - \beta \ket{1}$ &  $Z$  & $\alpha \ket{0} + \beta \ket{1}$ \\  
 11  & $\alpha \ket{1} - \beta \ket{0}$ &  $ZX$ & $\alpha \ket{0} + \beta \ket{1}$ \\
\hline
\end{tabular}

\vskip5pt

{\small
\href{http://www.rec.vsu.ru/rus/ecourse/quantcomp/sem8.pdf}
     {``Квантовые компьютеры'', Семинар 8, алг.Дойча}

\href{http://www.rec.vsu.ru/rus/ecourse/quantcomp/sem10.pdf}
     {``Квантовые компьютеры'', Семинар 10, телепортация}
\\
%\vskip5pt
  \href{https://mipt.ru/science/labs/QIT-lab/for-students.php}
     {С. Н. Филиппов, глава 13}
}
}

\frame {
  \frametitle{Codeforces / Microsoft}

  \url{https://codeforces.com/contest/1356}
  {Соревнование лето 2020 и тренировка}
\\
\vskip5pt

  \href{https://codeforces.com/blog/entry/77614}
       {Анонс соревнования 2020}
\\
  Соревнования Summer 2018 и Winter 2019, вместе с тренировками и решениями прошлых задач.
}

\frame {
  \frametitle{Перспективы}

  \href{https://docs.microsoft.com/en-us/learn/modules/solve-graph-coloring-problems-grovers-search/}
  {Graph coloring with Grover's search}
\\
  \href{https://github.com/epiqc/ScaffCC/}
  {Scaffold компилятор на основе LLVM}
\\
  \href{https://arxiv.org/pdf/1207.0511v5.pdf}
  {Модульное умножение для алг.Шора}
\\
  \href{https://arxiv.org/abs/quant-ph/0407095}
  {Эллиптические кривые для DLOG}
\\
  \href{https://arxiv.org/abs/2001.04383}
  {mip*=re}
}

\end{document}

